\documentclass[a4paper, 12pt, article]{memoir}
\usepackage[utf8]{inputenc}
\usepackage{hyperref}
\usepackage{indentfirst}
\usepackage{listings}
\lstset{language=C++}
\lstset{keepspaces=true}
\lstset{frame=lines}
\lstset{numbers=left}
\usepackage{caption}
\captionsetup[table]{name=Tabela}
\usepackage[brazilian]{babel}
\renewcommand{\appendixname}{Anexo} %trocar Apendice por Anexo


\title{Nova - Gramática}
\author{Rubens Pessoa}
\date{\today}
\newcommand{\institution}{Universidade Federal de Alagoas
  \\ Instituto de Computação}
\newcommand{\department}{Ciência da Computação}

\renewcommand{\maketitlehooka}{
  \centering
  \institution\\
  \emph{\department}\\[.2cm]
  \par
  \hrulefill
  \vfill}
\renewcommand{\maketitlehookb}{\vfill}

\renewcommand{\thesection}{\arabic{section}}
\renewcommand{\thesubsection}{\thesection.\arabic{subsection}}
\makeatletter
\let\l@subsection\l@section
\let\l@section\l@chapter
\makeatother
\setsecnumdepth{subsection}

\begin{document}

	\include{anexosintaxe}
	\include{anexolog}

\frontmatter
\begin{titlingpage}
  \maketitle
\end{titlingpage}

\tableofcontents

\mainmatter

\section{Especificação da Gramática da Linguagem}
\label{sec:intro}
O Analisador Sintático a ser utilizado para a linguagem de programação NOVA será o Analisador Descendente Preditivo Recursivo. Este Analisador requer que a gramática da linguagem seja convertida para a gramática LL(1), o que será demonstrado nas próximas seções.

\newpage

\section{Gramáticas}


\end{document}